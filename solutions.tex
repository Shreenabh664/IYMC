\documentclass{article}
\author{Shreenabh Agrawal}
\title{IYMC Pre-Final Round 2020}


\usepackage{amsmath}
\usepackage{amssymb}
\usepackage{graphicx}
\usepackage{mathtools}
\usepackage{siunitx}

\begin{document}

\maketitle
\newpage

\section*{Problem A.1}
Find all points $(x, y)$ where the functions $f(x), g(x), h(x)$ have the same value:
$$
f(x)=2^{x-5}+3, \quad g(x)=2 x-5, \quad h(x)=\frac{8}{x}+10
$$
\section*{Solution A.1}
Solving for $g(x)$=$h(x)$,
$$2x-5=\frac{8}{x}+10$$
Multiply by $x$, ($x\neq0$)

\begin{align*}
2x^2-15x-8&=0 \\
2x^2-16x+x-8&=0 \\
2x(x-8)+1(x-8)&=0\\
(2x+1)(x-8)&=0
\end{align*}
$$\boxed{x=8, x=-\frac{1}{2}}$$
$f\left(-\frac{1}{2}\right)$ is irrational while $g\left(-\frac{1}{2}\right)$ and $h\left(-\frac{1}{2}\right)$ are rational. Hence, they cannot be equal.

\begin{align*}
    f(8)&=2^{8-5}+3=11, \\
    g(8)&=2\times8-5=11, \\
    h(8)&=\frac{8}{8}+10=11
\end{align*}
$\therefore$ (8,11) is the only point where $f(x), g(x), h(x)$ have the same value.

\newpage
\section*{Problem A.2}
Determine the roots of the function $f(x)=\left(5^{2 x}-6\right)^{2}-\left(5^{2 x}-6\right)-12$.
\section*{Solution A.2}
Let $5^{2x}-6$ be $t$. We have:
\begin{align*}
    t^2-t-12&=0 \\
    t^2-4t+3t-12&=0 \\
    t(t-4)+3(t-4)&=0 \\
    (t+3)(t-4)&=0
\end{align*}
$$\boxed{t=-3, t=4}$$
\subsection*{Case I}
Let $t=-3$,
\begin{align*}
    5^{2x}-6&=-3 \\
    5^{2x}&=3 \\
    2x &= \log_5{3} \\
    x &= \frac{1}{2}\log_5{3}
\end{align*}
\subsection*{Case II}
Let $t=4$,
\begin{align*}
    5^{2x}-6&=4 \\
    5^{2x}&=10 \\
    2x&=\log_5{10} \\
    x&=\frac{1}{2}\log_5{10}
\end{align*}
$$\boxed{x=\frac{1}{2}\log_5{3}, \frac{1}{2}\log_5{10}}$$

\newpage
\section*{Problem A.3}
Find the derivative $f_{m}^{\prime}(x)$ of the following function with respect to $x$ :
$$
f_{m}(x)=\left(\sum_{n=1}^{m} n^{x} \cdot x^{n}\right)^{2}
$$
\section*{Solution A.3}
By chain rule of derivatives we have,
$$f(g(x))=f'(g(x))g'(x)$$
\begin{align*}
    f_m(x)&=\left(\sum\limits_{n=1}^{m}n^x\cdot x^n\right)^2 \\
    f'_m(x)&=2\left(\sum\limits_{n=1}^{m}n^x\cdot x^n\right)\left(\sum\limits_{n=1}^{m}n^x\cdot x^n\right)'
\end{align*}
Now by product rule of derivatives we have,
\[\left(f(x)g(x)\right)'=f'(x)g(x)+f(x)g'(x)\]
\[f'_m(x)=2\left(\sum_{n=1}^{m}n^x\cdot x^n\right)\left(\sum_{n=1}^{m} \left(n^x \cdot x^n \cdot \ln(n) + n^{x+1}x^{n-1}\right)\right)\]
\[\boxed{f'_m(x)=2\left(\sum_{n=1}^{m}n^x\cdot x^n\right)\left(\sum_{n=1}^{m}n^x \cdot x^{n-1} (x \ln(n)+n)\right)}\]

\newpage
\section*{Problem A.4}
Find at least one solution to the following equation:
$$
\frac{\sin \left(x^{2}-1\right)}{1-\sin \left(x^{2}-1\right)}=\sin (x)+\sin ^{2}(x)+\sin ^{3}(x)+\sin ^{4}(x)+\cdots
$$
\section*{Solution A.4}
As $|\sin(x)| \leq 1$, R.H.S. can be written as sum of infinite geometric progression as:
\[\frac{\sin(x^{2}-1)}{1-\sin(x^{2}-1)}=\frac{\sin(x)}{1-\sin(x)}\]
One solution can be (in principal interval):
\begin{align*}
    \sin(x^{2}-1)&=\sin(x) \\
    x^{2}-1&=x
\end{align*}
Where, $-\pi \leq x^{2}-1 \leq \pi \implies x \in [-\sqrt{\pi+1}, \sqrt{\pi+1}]$,
\[x^2-x-1=0\]
By quadratic formula,
$x=\frac{1\pm \sqrt{1+4}}{2} \implies \frac{1-\sqrt{5}}{2}$
This satisfies the initial conditions also. Thus we have,
\[\boxed{x=\frac{1-\sqrt{5}}{2}}\]

\newpage
\section*{Problem B.1}
Consider the following sequence of successive numbers of the $2^{k}$ -th power:
$$
1,2^{2^{k}}, 3^{2^{k}}, 4^{2^{k}}, 5^{2^{k}}, \ldots
$$
Show that the difference between the numbers in this sequence is odd for all $k \in \mathbb{N}$.
\section*{Solution B.1}
Natural powers of odd numbers are odd and natural powers of even numbers are even. This is because raising a number to its power does not introduce/remove new factors of 2. Hence parity is retained. This applies to $2^{k}$-th powers too. \\
Thus, the sequence will be like:
\[1, even, odd, even, odd, even, \ldots\]
Here, as we can observe, difference of two consecutive terms is always odd. This is because:
\[even-odd = odd\]
\[odd-even = odd\]
Hence proved, the difference between the numbers in this sequence is odd for all $k \in \mathbb{N}$.

\newpage
\section*{Problem B.2}
Prove this identity between two infinite sums (with $x \in \mathbb{R}$ and $n !$ stands for factorial):
$$
\left(\sum_{n=0}^{\infty} \frac{x^{n}}{n!}\right)^{2}=\sum_{n=0}^{\infty} \frac{(2 x)^{n}}{n!}
$$
\section*{Solution B.2}
We know the Taylor series expansion:
\[e^{x}=\left(\sum_{n=0}^{\infty} \frac{x^{n}}{n !}\right)\]
Squaring both sides we have,
\[e^{2x}=\left(\sum_{n=0}^{\infty} \frac{x^{n}}{n !}\right)^{2}\]
But by replacing $x$ with $2x$ in the original Taylor series expansion, we can write $e^{2x}$ as:
\[e^{2x}=\sum_{n=0}^{\infty} \frac{(2 x)^{n}}{n!}\]
Hence proved:
\[\boxed{\left(\sum_{n=0}^{\infty} \frac{x^{n}}{n!}\right)^{2}=\sum_{n=0}^{\infty} \frac{(2 x)^{n}}{n!}}\]

\newpage
\section*{Problem B.3}
You have given a function $\lambda: \mathbb{R} \rightarrow \mathbb{R}$ with the following properties $(x \in \mathbb{R}, n \in \mathbb{N})$
$$
\lambda(n)=0, \quad \lambda(x+1)=\lambda(x), \quad \lambda\left(n+\frac{1}{2}\right)=1
$$
Find two functions $p, q: \mathbb{R} \rightarrow \mathbb{R}$ with $q(x) \neq 0$ for all $x$ such that $\lambda(x)=q(x)(p(x)+1)$
\section*{Solution B.3}
In this problem, we make use of the "Fractional Part Function" - \{x\}. \\
We have $\lambda(n)=0$ and \{n\} = 0 (by definition of F.P.F). So let us designate $q(x)=\{x\}$. \\
Next, we have $\lambda(x+1)=\lambda(x)$, this implies both $q(x)$ and $p(x)$ are purely composed of $\{x\}+c$ where $c$ is a numerical constant. We have already proved that in $q(x)$, $c=0$. \\
We are given: $\lambda\left(n+\frac{1}{2}\right)=1$. We also know that, $\left\{n+\frac{1}{2}\right\}=\left\{\frac{1}{2}\right\}$. Hence:
\begin{align*}
    \lambda\left(n+\frac{1}{2}\right)&=\left\{n+\frac{1}{2}\right\}\left(\left\{n+\frac{1}{2}\right\}+c+1\right)=1\\
    \frac{1}{2}\left(\frac{1}{2}+c+1\right)&=1 \\
    \frac{3}{4}+\frac{c}{2}&=1 \\
    c&=\frac{1}{2}
\end{align*}
\[\boxed{\therefore q(x)=\{x\}, p(x)=\{x\}+0.5}\]

\newpage
\section*{Problem B.4}
You have given an equal sided triangle with side length $a$. A straight line connects the center of the bottom side to the border of the triangle with an angle of $\alpha .$ Derive an expression for the enclosed area $A(\alpha)$ with respect to the angle (see drawing). \\
\includegraphics[scale=0.166666]{IM1.png}
\qquad
\includegraphics[scale=0.166666]{IM2.png}
\section*{Solution B.4}
We do the constructions and labelling as shown (two cases dealt separately).
\includegraphics[scale=0.25]{IMG1.png}
\qquad
\includegraphics[scale=0.26]{IMG2.png}
\subsection*{Case I}
In $\Delta DEC$,
\begin{align*}
    \tan(\ang{60})&=\frac{EF}{FC}=\frac{h}{p} \\
    h&=\sqrt{3}p \\
    \tan(\alpha)&=\frac{EF}{DF} \\
                &=\frac{h}{\frac{a}{2}-p} \\
                &=\frac{\sqrt{3}p}{\frac{a}{2}-p} \\
    \frac{a}{2}\tan(\alpha)-p\tan(\alpha)&=\sqrt{3}p \\
    \frac{a}{2}\tan(\alpha)&=(\sqrt{3}+\tan(\alpha))p \\
    p&=\frac{a \tan(\alpha)}{2(\sqrt{3}+\tan(\alpha))} \\
    \therefore h&= \sqrt{3}p \\
                &= \frac{\sqrt{3}a \tan{\alpha}}{2(\sqrt{3}+\tan(\alpha))} \\
    \therefore Ar(\Delta DEC) (=\Delta = A(\alpha)) &= \frac{1}{2} \cdot \frac{a}{2} \cdot h \\
\end{align*}
$$\boxed{A(\alpha)=\frac{\sqrt{3} a^{2} \tan(\alpha)}{8 (\sqrt{3}+\tan(\alpha))}}$$
Note: This result is valid only for $\ang{0} \leq \alpha < \ang{90}$.
\subsection*{Case II}
\begin{align*}
    \alpha + \beta &= \ang{180} \\
    \Delta + \Delta' &= \frac{\sqrt{3}a^{2}}{4}
\end{align*}
From our result in Case-I,
\[\Delta' = \frac{\sqrt{3} a^2 \tan(\beta)}{8(\sqrt{3}+\tan(\beta))}\]
Substituting $\beta = \ang{180} - \alpha$,
\begin{align*}
    \Delta' &= \frac{\sqrt{3}a^{2}\tan(\ang{180}-\alpha)}{8(\sqrt{3}+\tan(\ang{180}-\alpha))} \\
            &= \frac{\sqrt{3}a^{2}\tan(\alpha)}{8(\tan(\alpha)-\sqrt{3})}
\end{align*}
Using the sum of $\Delta$ and $\Delta'$,
\begin{align*}
    \Delta + \frac{\sqrt{3}a^{2}\tan(\alpha)}{8(\tan(\alpha)-\sqrt{3})} &= \frac{\sqrt{3}a^{2}}{4} \\
    \Delta &= \frac{\sqrt{3}a^2}{4}\left\{1-\frac{\tan(\alpha)}{2(\tan(\alpha)-\sqrt{3})}\right\}
\end{align*}
\[\boxed{A(\alpha)=\frac{\sqrt{3}a^2}{8}\cdot \frac{2\sqrt{3}-\tan(\alpha)}{\sqrt{3}-\tan(\alpha)}}\]
This result is valid only for $\ang{90} < \alpha \leq \ang{180}$.
\medskip

\noindent
For $\alpha = \ang{90}$, the shaded area is half of the total area and equal to $\frac{\sqrt{3}a^2}{8}$. This can also be shown by evaluating the limit as $\alpha \rightarrow \ang{90}$ in any of the formulae derived in the cases above.
\medskip

\noindent
To sum up,
\[\boxed{
    A(\alpha)=
\begin{dcases}
    \frac{\sqrt{3}a^2 \tan(\alpha)}{8(\sqrt{3}+\tan(\alpha))},           & \ang{0} \leq \alpha < \ang{90} \\
    \frac{\sqrt{3}a^2}{8},                                               & \alpha = \ang{90} \\
    \frac{\sqrt{3}a^2(2\sqrt{3}-\tan(\alpha))}{8(\sqrt{3}-\tan(\alpha)}, & \ang{90} < \alpha \leq \ang{180}
\end{dcases}
}
\]
\end{document}
